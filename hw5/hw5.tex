\documentclass[10pt]{article} 
\usepackage[T1]{fontenc} 
\usepackage[utf8]{inputenc}
\usepackage{geometry} 
\geometry{verbose,marginparwidth=0.5in,tmargin=1in,bmargin=1in,lmargin=1in,rmargin=1in} 
\usepackage{lmodern}

\usepackage{booktabs}

\usepackage{enumitem}
% \setlist{nosep}

% \usepackage{amsfonts}
% \usepackage{amsmath}
\usepackage{comment}

\usepackage{mathtools}
\usepackage{bbm}
\newcommand{\one}{\mathbbm{1}}
\newcommand{\bs}{\boldsymbol}


\begin{document}


\begin{center}
\textbf{Econ 512}

\emph{Fall 2018}\\[1em]

Homework 5 -- Binary Choice MLE with Random Coefficients\\
Due 11/28/2018 \\[3em]
\end{center}


Consider the following binary discrete choice model for panel data:
\begin{equation*}
Y_{it}=I\left( \beta _{i}X_{it}+\gamma Z_{it}+u_{i}+\epsilon _{it}>0\right) ,
\end{equation*}
where $X_{it}$ and $Z_{it}$ are scalar regressors for a group of $i=1,...,N$ individuals and $t=1,...,T$ time periods. $\ N=100$ and $T=20$. \ The coefficients $\beta _{i}$ and $u_{i}$ are person specific and are modeled as draws from a bivariate normal distribution:

\begin{equation*}
\begin{bmatrix}
\beta _{i} \\ 
u_{i}%
\end{bmatrix}%
\sim N(\mu ,\Sigma ),\quad \text{where}\quad \mu =%
\begin{bmatrix}
\beta _{0} \\ 
u_{0}%
\end{bmatrix}%
,\quad \Sigma =%
\begin{bmatrix}
\sigma _{\beta } & \sigma _{\beta u} \\ 
\sigma _{\beta u} & \sigma _{u}%
\end{bmatrix}%
.
\end{equation*}

Assume $\epsilon _{it}$ follows standard logistic distribution, i.e. $%
F(\epsilon )=\left( 1+e^{-\epsilon }\right) ^{-1}$, and then a single
contribution to the likelihood function from individual $i$ is

\begin{equation*}
L_{i}(\gamma \mid \beta _{i},u_{i})=\prod_{t=1}^{T}F(\beta _{i}X_{it}+\gamma
Z_{it}+u_{i})^{Y_{it}}\left[ 1-F(\beta _{i}X_{it}+\gamma Z_{it}+u_{i})\right]
^{1-Y_{it}}.
\end{equation*}

To construct the likelihood function for the data set of NT observations we
have to integrate over the joint distribution of $(\beta _{i},u_{i})$. \ The
likelihood function for the data set is: 
\begin{equation*}
L(\gamma ,\mu ,\Sigma )=\prod_{i=1}^{N}\int_{-\infty }^{\infty
}\int_{-\infty }^{\infty }L_{i}(\gamma \mid \beta _{i},u_{i})\phi (\beta
_{i},u_{i}\mid \mu ,\Sigma )\,\mathrm{d}\beta _{i}\,\mathrm{d}u_{i},
\end{equation*}%
where $\phi (\cdot \mid \mu ,\Sigma )$ is the joint density function of
bivariate normal distribution $N(\mu ,\Sigma )$.  Of course, it will be numerically more convenient to work with the log-likelihood function. 
The data set {\tt hw5.mat} contains 20 x 100 matrices of the variables $X,Z$, and $Y$.

\begin{enumerate}
	\item Assume $u_i=0\,\,\forall\,i$ (ie. take $u_i$ out of the model, so that $u_0=\sigma_u=\sigma_{u\beta}=0$). Use Gaussian Quadrature using 20 nodes to calculate the log-likelihood function when $\beta_0=0.1$, $\sigma_{\beta}=1$, and $\gamma=0$. \\[1em]
	The calculated log-likelihood function is  -1.5345e+03. \\[1em]
	\item Now use Monte Carlo Methods using 100 nodes to calculate the log-likelihood function. \\[1em]
     The calculated log-likelihood function is  -1.2385e+03 \\[1em]
	\item Maximize (or minimize the negative) log-likelihood function with respect to the parameters using both integration techniques above. Use Matlab's fmincon without a supplied derivative to max (min) your objective function.  \\[1em]
	Initial value is para=[$\beta$, $\sigma_\beta$, $\gamma$]=[0.1; 2; 1] \\
	The results using Guassian Quadrature are:\\
	para1 = [0.4772, 0.0147, 0.0177]      max1 = -1.4534e+03 \\
	The results using Monte Carlo Methods are:\\
    para2 = [1.4749, 2.5202, 0.4761]     max2 = -2.6987e+03  \\[1em]
	\item Now allow $u_0\ne0$, so allow the parameters $\sigma_u$ and $\sigma_{\beta u}$ to be non-zero. Maximize the log-likelihood function, estimating all of the parameters, using Monte Carlo methods.\\[1em]
	Initial value is para=[$\beta_0$, $u_0$, $\sigma_\beta$, $\sigma_{\beta u}$, $\sigma_u$, $\gamma$]= [1; 1; 1; 0.5; 1; 0.5] \\
	The results are: \\
	para3 =[1.0001; 1.0003; 1.0000; 0.5002; 0.9998; 0.5004] max3 = -1.6338e+03 \\[1em]
	\item For each estimation, report the starting value, argmax, and maximized value of the log-likelihood function.
\end{enumerate}

\vspace{2em}
(Hint: the matlab function ``chol'' may come in handy for simulating from the joint density.) 



\end{document}