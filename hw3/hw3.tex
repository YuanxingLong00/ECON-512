
\documentclass{article}
%%%%%%%%%%%%%%%%%%%%%%%%%%%%%%%%%%%%%%%%%%%%%%%%%%%%%%%%%%%%%%%%%%%%%%%%%%%%%%%%%%%%%%%%%%%%%%%%%%%%%%%%%%%%%%%%%%%%%%%%%%%%%%%%%%%%%%%%%%%%%%%%%%%%%%%%%%%%%%%%%%%%%%%%%%%%%%%%%%%%%%%%%%%%%%%%%%%%%%%%%%%%%%%%%%%%%%%%%%%%%%%%%%%%%%%%%%%%%%%%%%%%%%%%%%%%
\usepackage{geometry}


\newtheorem{theorem}{Theorem}
\newtheorem{acknowledgement}[theorem]{Acknowledgement}
\newtheorem{algorithm}[theorem]{Algorithm}
\newtheorem{axiom}[theorem]{Axiom}
\newtheorem{case}[theorem]{Case}
\newtheorem{claim}[theorem]{Claim}
\newtheorem{conclusion}[theorem]{Conclusion}
\newtheorem{condition}[theorem]{Condition}
\newtheorem{conjecture}[theorem]{Conjecture}
\newtheorem{corollary}[theorem]{Corollary}
\newtheorem{criterion}[theorem]{Criterion}
\newtheorem{definition}[theorem]{Definition}
\newtheorem{example}[theorem]{Example}
\newtheorem{exercise}[theorem]{Exercise}
\newtheorem{lemma}[theorem]{Lemma}
\newtheorem{notation}[theorem]{Notation}
\newtheorem{problem}[theorem]{Problem}
\newtheorem{proposition}[theorem]{Proposition}
\newtheorem{remark}[theorem]{Remark}
\newtheorem{solution}[theorem]{Solution}
\newtheorem{summary}[theorem]{Summary}
\newenvironment{proof}[1][Proof]{\noindent\textbf{#1.} }{\ \rule{0.5em}{0.5em}}
\geometry{left=1in,right=1in,top=1in,bottom=1in}

\begin{document}

\begin{center}
\textbf{Econ 512}

\emph{Fall 2018}\\[1em]

Homework 3 -- Nonlinear Optimization \\
Due 10/10/2018
\\[3em]
\end{center}

\bigskip


In 1969, the popular magazine \textit{Psychology Today }published a
101-question survey on affairs. Professor Ray Fair (1978) extracted a sample
of 601 observations on men and women who are currently married for the first
time and analyzed their responses to a question about extramarital affairs.
He used the tobit model as his estimation framework for this study. The
dependent variable is a count of the number of affairs which suggests that a
standard Poisson model may be a better choice. \ Download the data set
hw3.mat, and estimate the parameters by the methods of nonlinear least
squares and maximum likelihood using different algorithms.

Data description:

$y$ - count data: number of affairs in the past year

$\mathbf{x}$ - constant term=1, age, number of years married, religiousness
(scale $1-5$), occupation (scale $1-7$), self-rating of marriage (scale $1-5$%
)

\bigskip

The data generating assumptions for the Poisson model, where $j$= number of
affairs, are:\bigskip

\begin{eqnarray*}
\Pr \left[ y_{i}=j\right] &=&\frac{e^{-\lambda _{i}}\lambda _{i}^{j}}{j!} \\
\log \lambda _{i} &=&\mathbf{x}_{i}^{\prime }\mathbf{\beta } \\
E\left( y_{i}\left\vert x_{i}\right. \right) &=&e^{\mathbf{x}_{i}\prime 
\mathbf{\beta }}
\end{eqnarray*}%
for some $\mathbf{\beta }=\left( \beta _{0},\beta _{1},\beta _{2},\beta
_{3},\beta _{4},\beta _{5}\right) ^{\prime }.$

\bigskip

The log-likelihood function is:

\begin{eqnarray*}
\ln L &=&\sum\limits_{i=1}^{n}\ln f\left( y_{i}\left\vert x_{i}\right.
,\beta \right) \\
&=&\sum\limits_{i=1}^{n}\ln \frac{e^{-\lambda _{i}}\lambda _{i}^{j}}{j!} \\
&=&\sum\limits_{i=1}^{n}\left[ -\lambda _{i}+y_{i}\ln \lambda _{i}-\ln j!%
\right] \\
&=&\sum\limits_{i=1}^{n}\left[ -e^{\mathbf{x}_{i}\prime \mathbf{\beta }%
}+y_{i}x_{i}^{\prime }\beta -\ln y_{i}!\right]
\end{eqnarray*}

The residual sum of squares is:%
\[
S\left( \beta \right) =\sum\limits_{i=1}^{n}\left( y_{i}-e^{\beta ^{\prime
}x_{i}}\right) ^{2} 
\]

\begin{enumerate}
\item Estimate the parameter vector $\beta$ using the maximum likelihood estimator computed via  
the Nelder-Mead simplex method. \\[2em]

Please check my matlab codes for the algorith. The MLE estimator is [2.5924, -0.0333, 0.1162, -0.3572, 0.0783, -0.4131] when starting value[1,1,1,1,1,1] and the Nelder-Mead simplex method are used. \\[3em]






\item Estimate the parameter vector $\beta$ using the maximum likelihood estimator computed via  
a quasi-Newton optimization method, report which method you choose.\\[2em]

I use Broyden's method to solve for MLE. First, I find the FOC: 
\begin{equation}
\sum_{i=1}^n  \lbrace -x_i e^{x'_i\beta}+y_i x_i \rbrace=0
\end{equation}
The Jacobian of equation (1) is: $$\sum_{i=1}^n  \lbrace -x_i x'_ie^{x'_i\beta} \rbrace $$
The estimator generated using the estimator in the above question as initial value, is [2.5339, -0.0323, 0.1157, -0.3540, 0.0798, -0.4094].    \\[3em]


 

\item Estimate the parameter vector $\beta$ using nonlinear least squares estimator computed using
the command {\tt lsqnonlin}. What computation method are you using? \\[2em]

I use the trust-region-reflective option in solver {\tt lsqnonlin} and the inital value [2.5, -0.0, 0.1, -0.4, 0.1, -0.5]. The NLS estimator got is [2.5122, -0.0384, 0.1141, -0.2799,  0.0677, -0.3697]. \\[3em]


\item Estimate the parameter vector $\beta$ using the nonlinear least squares estimator computed using 
the Nelder-Mead simplex method. \\[2em]

The estimator is [2.5339   -0.0323    0.1157   -0.3540    0.0798   -0.4094] in this case. \\[3em]


\item Test all four approaches with regard to the choice of initial values.  Roughly rank them in order of 
robustness and time to convergence. Submit a short writeup summarizing your results.  \\[2em]

I select 100 starting points  and test the four approaches.\\

For the first method, number of convergence is 32.Conditional on convergence, the average time to convergence is 0.2414 second.\\
For the second method, number of convergence is 4, when we select the Broyden iteration number to be 100. And the average time to convergence is 0.024 second.\\
For the third method, number of convergence is 91. And the average time to convergence is 0.0699 second.\\
The fourth approach is the same as the first approach when specifying the starting value. \\
Therefore, the most robust method is Nonlinear least sqaure when using the lsqnonlin solver. And conditional on convergence, the Broyden's method would be faster. 








\end{enumerate}


\end{document}
